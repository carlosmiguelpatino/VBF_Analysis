\chapter{The Neutrino Mass}

The Standard Model predicts that the neutrino is a massless particle. However, experiments like the ones mentioned in chapter \ref{chap: Introduction} have proven that neutrino oscillations exist. These oscillations show that, contrary to the predictions of the Standard Model, neutrinos are particles with mass. In this chapter the reasons for the predictions of zero mass for the neutrino in the Standard Model are given. Also, in this chapter some the formalism of the Type I See Saw mechanism is included.  

\section{Neutrino Mass in the Standard Model}

Fermions in the Standard Model have Dirac mass terms of the form $m\bar{\psi}\psi$ in the total Langrangian. If the field $\psi$ is decomposed into the corresponding left and right chiral states, the Dirac mass term can be written as 

$$m(\bar{\psi_{L} + \psi{R}})(\psi_{L} + \psi_{R})$$

As shown in Appendix \ref{app: samechirality}, the terms $\bar{\psi_{R}}\psi_{R}$ and $\bar{\psi_{L}}\psi_{L}$ are zero. Therefore, the Dirac mass term can be written as shown in Equation \ref{eq: massterm}. As mentioned in Chapter \ref{chap: Introduction}, the Standard Model only includes a neutrino with left chirality. The abscence of a neutrino with right chirality makes that a neutrino cannot have mass from a Dirac mass term, therefore in the Standard Model it needs to have mass zero.

\begin{equation} \label{eq: massterm}
m \left(\bar{\psi_{L}}\psi_{R} + \bar{\psi_{R}}\psi_{L}\right)
\end{equation}

Another mechanism that could provide mass to the neutrino in the Standard Model is the Majorana mass term. The Majorana mass term has the form shown in Equation \ref{eq: majoranaMass}, where $\nu^{C}$ is defined as $\nu^{\intercal}C^{-1}$. $C$ is known as the charge conjugation matrix, because it inverts the charge of the field it acts on. 

\begin{equation}\label{eq: majoranaMass}
m \bar{\nu_{L}^{C}}\nu_{L}
\end{equation}

As seen in Equation \ref{eq: majoranaMass}, in principle the neutrino could have mass in the Standard Model including a Majorana mass term. However this term would not conserve the Lepton number, i.e. would violate the L symmetry, that is conserved throughout the Standard Model. This happens because the Majorana fermions, in this case neutrinos, are their own antiparticles. That is why the Lepton number of a Majorana neutrino would be either $L = 1$ or $L = -1$. That is why the Majorana mass term would violate de lepton conservation number by $\Delta L = \pm 2$.

Since these two mechanisms are not useful to explain the origin of the neutrino mass inside the Standard Model, a minimal extension to the SE is proposed in order to provide the neutrinos with mass. This extension consists in inserting right handed neutrinos to the model to explain the origin of the neutrino mass. This extension and its consequences is described in the next section.

\section{The See Saw Mechanism} 



 


