\section{Definitions} \label{sec:definitions}

\subsection{Variable Definitions}

The transverse momentum or $p_{T}$, is defined as the momentum component that a particle has in the plane perpendicular to the beam line. In the coordinate system of the LHC, this plane corresponds to the $x-y$ plane.

The variable related with the polar angle in the LHC is called pseudorapidity, or $\eta$, defined as in Equation \ref{eq: eta}. The use of this variable is justified for mainly two reasons. The first one is that $\Delta \eta$, contrary to $\Delta \theta$, is a Lorentz invariant. This makes $\Delta \eta$ a more natural variable than $\Delta \theta$ for relativistic calculations. The second reason is that the distribution of the values of $\eta$ in barrel region, where the multiplicity of particle is less than in the end-caps, is wider allowing the $\eta$ particle distribution to be aproximately constant.

\begin{equation}
 \eta = -\ln\left[\tan\left(\frac{\theta}{2}\right)\right]
 \label{eq: eta}
\end{equation}

In the preliminary analysis of the variable distributions shape using normalized to the unit plots, i.e. the area under the distribution for the distribution is equal to one, a separation between signal and background was achieved in the transverse momentum variable of both jets and taus. However, in distribution involving all the cuts considered fothe analysis, the separation noticeable but considerably smaller. The separation between background and signal was bigger for the tau transverse momentum $p_{T}(\tau)$ than the one shown by both leading and sub-leading jets.

With the idea of exploiting the small but existent different between signal and background in the $p_{T}$ varibles, two new variables shown in Equations \ref{eq: HT} and \ref{eq: ST} were created to check for possible further separation between signal and background. As shown in equation \ref{eq: HT}, the $H_{T}$ variable is defined as the scalar sum of the jets with $p_{T}$ greater than 30 GeV and $|\eta| < 5$ that are not B-jets. $S_{T}$ is defined as the scalar sum of jets that fullfilll the same conditions of $H_{T}$, added to the $p_{T}$ of the $\tau$'s in the event.

Since the $\tau$ selection is important for this analysis, it is relevant to provide a further description of the selection criteria for the $\tau$'s in the simulated events. For starters, a jet identified as a tau is considered a valid $\tau$ if it has minimum a transverse momentum of 20 GeV. Also, it was required that a valid $\tau$ should not overlap with an electron or a muon. That is, the $\Delta R$, defined as $\Delta R = \sqrt{(\Delta \eta)^2 + (\Delta \phi)^2}$ should not be less than 0.3. This condition guarantees that the jet identified as a $\tau$ does not overlap with other leptons. Other condition required for a valid $\tau$ is that the jet has a minimum transverse momentum of 20 GeV. Since the final state for this analysis includes two $tau$'s, the two taus with greater $p_{T}$ are selected among a maximum of three taus stored for each event. The leading $\tau$ is the one with highest $p_{T}$ and the sub-leading $\tau$ is the one with second highest $p_{T}$.

\begin{equation}
 H_{T} = \sum_{i=1}^{n} p_{T}(jet_{i})
 \label{eq: HT}
\end{equation}

\begin{equation}
 S_{T} = \sum_{i=1}^{n} p_{T}(jet_{i}) + \sum_{j=1}^{m} p_{T}(\tau_{j})
 \label{eq: ST}
\end{equation}

\subsection{Cut Definitions}

In order to achieve a separation between background and signal, several successive cuts in variables were made. This section contains a list of the cuts with its explanation. 
