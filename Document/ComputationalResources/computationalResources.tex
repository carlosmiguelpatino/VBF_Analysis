\chapter{Computational Resources}

The project requires computational work, because simulations of events from the different processes are needed. Also, an analysis of the samples using the analysis code is required. The background and signal samples will be simulated using the software MadGraph \cite{MadGraph}, Pythia \cite{Pythia} and Delphes \cite{Delphes}. The data analysis and all the subsequent kinematic, topological, and optimal cuts analyses will be performed using ROOT software \cite{ROOT}.

MadGraph is an event generator software that allows the simulation of collision between two particle beams. For this analysis in particular, the simulations will consist in proton collision at 13 TeV in order to reproduce the actual conditions of the LHC. MadGraph includes the physical parameters that determine the production probability of a given process, as well as the possible decays of the simulated particles. Besides providing the necessary matrices to calculate the cross sections of the processes, MadGraph also creates the pictorial representations of the Feynman Diagrams from the generated processes. To this end, the software uses perturbation theory in the calculations of production and generation of physical processes.

Pythia is a software that allows the simulation of various strong processes models that evolve from a few bodies to final states with high particle multiplicity. Particularly, in this case Pythia will be used for the simulation of quark and gluon fragmentation processes. This fragmentation process occurs when, due to and intrinsic characteristic of the strong interaction, there is an energy gain caused by the increase of the distance of two bound quarks. If the separation is enough to reach a critical energy, a pair quark-antiquark is created. The Pythia simulation is necessary, because processes like the ones mentioned above occur during a proton collision at the LHC.

Delphes is a software used to add the effects that a multipurpose detector, like ATLAS or CMS, may have on the particles to the Monte Carlo simulations performed for different processes. In this particular case, Delphes is necessary to simulate the interaction of the particles coming from the generated processes in MadGraph and Pythia with the CMS components. Namely, reproducing the conditions of the detector and the uncertainties coming from the measuring process is achieved by using Delphes. The changes in the cinematic variables due to their interaction with matter, errors caused by the electronics of the detector, and the additional particles generated because of the interaction between the particles and the detector components can be accounted for using Delphes. Other functionalities included in Delphes are: simulation of the detector geometry, the effect of the magnetic field over the particles, and the particle identification and reconstruction efficiencies, among others.

ROOT is a software library developed by CERN to perform data analyses related with particle physics. One of the main characteristics of this library is the possibility of handling large volumes of data efficiently. The latter is achieved by using a tree structure in which the information related with the particles is stored and can be accessed easily using ROOT functionalities. Other features included in the library are the creation of histograms from data trees, multivariate analysis, four-vector calculations, among others. By using ROOT functionalities, it is also possible to estimate optimal cuts in variables to reduce experimental noise to its minimum. This is why the entire final analysis will involve using tools provided by ROOT.
